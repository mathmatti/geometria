\chapter{Spazi duali}
Tra le applicazioni lineari su uno spazio vettoriale, sono particolarmente interessanti le \emph{forme} lineari, o \emph{funzionali} lineari, ossia le applicazioni da $V$ al campo $K$ su cui è definito.
Lo spazio (anch'esso vettoriale) dei funzionali lineari è detto \emph{spazio duale} di $V$, e si indica tradizionalmente con $V^*$.
Per quanto visto nei capitoli precedenti $V^*$ è isomorfo a $\mat(1,\dim V,K)$, in quanto questi funzionali sono associati, scelta una base di $V$, a delle matrici con una sola riga: vediamo subito allora che $\dim V^*=\dim V$.

Ecco alcuni esempi importanti di funzionali lineari.
\begin{itemize}
	\item In $\R^n$, possiamo definire un iperpiano come il luogo degli zeri di una funzione
		\begin{equation*}
			f(x_1,\dots,x_n)=a_1x_1+\cdots+a_nx_n-b
		\end{equation*}
		per dei coefficienti $a_i\in\R$ non tutti nulli e $b\in\R$.
		In generale, per uno spazio vettoriale $V$ qualunque, non disponiamo di una base canonica, quindi non possiamo trovare immediatamente i coefficienti $x_1,\dots,x_n$ dell'equazione $f(x_1,\dots,x_n)=0$.
		I funzionali lineari forniscono un modo per caratterizzare gli iperpiani senza bisogno di coordinate (cioè di una base), ossia in modo \emph{canonico}.
		Possiamo interpretare l'equazione come l'applicazione di un funzionale $\alpha$ del duale di $\R^n$ sul vettore $v=[x_1\,\cdots\, x_n]^T$, ossia come $\alpha(v)=b$.
		Per un generico spazio $V$ possiamo definire un iperpiano come l'insieme $\{v\in V\colon \alpha(v)=b\}$ per un dato funzionale $\alpha\in V^*$.
	\item Nello spazio delle funzioni continue reali in $[0,1]$, una mappa da esso a $\R$ può essere ad esempio 
		\begin{equation*}
			f\mapsto\int_0^1f(t)\,\dd t
		\end{equation*}
		che è evidentemente lineare.
		Possiamo definire un funzionale lineare anche a partire dalle funzioni stesse di $\cont{}\big([0,1]\big)$: ad una funzione $g$ di tale spazio associamo il funzionale
		\begin{equation*}
			T_g\colon f\mapsto\int_0^1f(t)g(t)\,\dd t.
		\end{equation*}
		Anche il funzionale definito precedentemente è di questo tipo, con $g=1$.
\end{itemize}

\section{Base duale}
\label{sec:base-duale}
Vediamo ora quelli che probabilmente sono i funzionali lineari più importanti di $V^*$.
Scegliamo una base $\mathcal B=\{b_1,\dots,b_n\}$ per lo spazio vettoriale $V$ sul campo $K$.
Sappiamo che un'applicazione lineare è univocamente determinata dalla sua azione su una base, dunque lo stesso vale per i funzionali lineari.
Definiamo $\beta_i\colon V\to K$ come
\begin{equation}
	\beta_i(b_j)=\delta_{ij}
	\label{eq:funzionale-coordinate}
\end{equation}
per $i,j=1,\dots,n$.
In altre parole, il funzionale $\beta_i$ ``estrae'' la $i$-esima coordinata di un vettore: se $v=\sum_{j=1}^nv_jb_j$ allora $\beta_i(v)=v_i$.

Consideriamo ora l'insieme $\mathcal B^*=\{\beta_1,\dots,\beta_n\}$: preso un qualsiasi $\alpha\in V^*$, possiamo scriverlo come combinazione lineare dei $\beta_i$ osservando che se $\alpha=\sum_{k=1}^n\alpha_k\beta_k$ allora
\begin{equation}
	\alpha(b_j)=\sum_{k=1}^n\alpha_k\beta_k(b_j)=\sum_{k=1}^n\alpha_k\delta_{kj}=\alpha_j
\end{equation}
e ciò identifica i coefficienti di $\alpha$ rispetto a $\mathcal B^*$.
Preso allora $v=\sum_{i=1}^nv_ib_i$, si ha
\begin{equation}
	\alpha(v)=\sum_{i=1}^nv_i\alpha(b_i)=\sum_{i=1}^nv_i\alpha_i
\end{equation}
e allo stesso tempo
\begin{equation}
	\sum_{i=1}^n\alpha_i\beta_i(v)=\sum_{i=1}^n\alpha_i\beta_i\bigg(\sum_{j=1}^nv_jb_j\bigg)=\sum_{i=1}^n\alpha_i\sum_{j=1}^nv_j\delta_{ij}=\sum_{i=1}^n\alpha_iv_i
\end{equation}
che coincide con la precedente.
Questo verifica che qualsiasi funzionale lineare può essere espresso come combinazione lineare dei $\beta_i$.
Inoltre, se $\sum_{i=1}^n\mu_i\beta_i=0$, cioè è il funzionale nullo, allora sulla base $\mathcal B$ risulta
\begin{equation}
	0=\sum_{i=1}^n\mu_i\beta_i(b_j)=\sum_{i=1}^n\mu_i\delta_{ij}=\mu_j
\end{equation}
per $j=1,\dots,n$ quindi $\mathcal B^*$ è anche linearmente indipendente: ciò prova che è una base di $V^*$.\footnote{
	Sapendo già che la dimensione di $V^*$ è $n$, l'indipendenza lineare di $\mathcal B^*$ sarebbe stata sufficiente a mostrare che è una base di $V^*$, ciononostante questa dimostrazione è utile per capire come individuare i coefficienti di un funzionale rispetto alla base $\mathcal B^*$.
}
\begin{definizione} \label{d:base-duale}
	Data una base $\mathcal B=\{b_1,\dots,b_n\}$ di uno spazio vettoriale $V$, la base $\{\beta_1,\dots,\beta_n\}$ di $V^*$ tale che per ogni $i,j=1,\dots,n$
	\begin{equation}
		\beta_i(b_j)=\delta_{ij}
	\end{equation}
	è detta \emph{base duale} di $\mathcal B$.
\end{definizione}
Come già visto, in questa base la $i$-esima componente di un funzionale lineare $\alpha$ è data da $\alpha(b_i)$.
Va detto che non ha senso parlare di una ``base duale di $V^*$'', dato che non è unica: ad ogni scelta di una base di $V$ corrisponde una base duale per $V^*$, a essa associata.

\section{Mappa duale}
Sia $T\colon V\to Z$ un'applicazione lineare tra gli spazi vettoriali $V$ e $Z$.
Definiamo la \emph{mappa duale} $T^*\colon Z^*\to V^*$ come
\begin{equation}
	T^*(f)=f\circ T.
	\label{eq:mappa-duale}
\end{equation}
Il risultato $T^*(f)$ è un funzionale lineare in $V^*$, detto anche \emph{pullback} di $f$ attraverso $T$.
Verifichiamo che è lineare: per $f,g\in Z^*$ e $\lambda,\mu\in K$ si ha per $v\in V$
\begin{equation}
	T^*(\lambda f+\mu g)(v)=(\lambda f+\mu g)\big(T(v)\big)=\lambda f\big(T(v)\big)+\mu g\big(T(v)\big)=\lambda T^*(f)(v)+\mu T^*(g)(v)
\end{equation}
che mostra la linearità di $T^*$.
\begin{teorema} \label{t:matrice-associata-trasposta}
	Siano $\mathcal B=\{e_1,\dots,e_n\}$ una base di $V$, $\mathcal C=\{g_1,\dots,g_m\}$ una base di $Z$, e $\mathcal B^*=\{\eta_1,\dots,\eta_n\}$ e $\mathcal C^*=\{\gamma_1,\dots,\gamma_m\}$ le rispettive basi duali di $V^*$ e $Z^*$.
	Se $A$ è la matrice associata a un'applicazione lineare $T\in\lin(V,Z)$ nelle basi $\mathcal B$ e $\mathcal C$, allora la matrice associata alla sua duale $T^*\in\lin(Z^*,V^*)$ nelle basi duali è la trasposta di $A$.
\end{teorema}
\begin{proof}
	La dimostrazione consiste semplicemente nel costruire le due matrici associate a $T$ e $T^*$.
	Se $A$ è la matrice associata a $T$ allora
	\begin{equation}
		T(e_i)=\sum_{j=1}^mA_{ji}g_j
	\end{equation}
	per $i=1,\dots,n$.
	Se $B$ è la matrice associata a $T^*$, allo stesso modo,
	\begin{equation}
		T^*(\gamma_i)=\sum_{j=1}^nB_{ji}\eta_j
	\end{equation}
	per $i=1,\dots,m$.
	Applicando $T^*(\gamma_i)$ agli elementi di $\mathcal B$ risulta cos\`i
	\begin{equation}
		T^*(\gamma_i)(e_k)=\sum_{j=1}^nB_{ji}\eta_j(e_k)=\sum_{j=1}^nB_{ji}\delta_{jk}=B_{ki}
	\end{equation}
	e d'altra parte per come è definita la mappa duale $T^*$ si ha
	\begin{equation}
		T^*(\gamma_i)(e_k)=\gamma_i\big(T(e_k)\big)=\gamma_i\bigg(\sum_{j=1}^mA_{jk}g_j\bigg)=\sum_{j=1}^mA_{jk}\gamma_i(g_j)=\sum_{j=1}^mA_{jk}\delta_{ij}=A_{ik}
	\end{equation}
	da cui $B_{ki}=A_{ik}$, ossia $B=A^T$.
\end{proof}

\begin{teorema} \label{t:iniettivita-suriettivita-mappa-duale}
	Siano $V$ e $Z$ due spazi vettoriali e $L\in\lin(V,Z)$.
	\begin{enumerate}
		\item Se $L$ è iniettiva, allora $L^*$ è suriettiva.
		\item Se $L$ è suriettiva, allora $L^*$ è iniettiva.
	\end{enumerate}
\end{teorema}
\begin{proof}
	\begin{enumerate}
		\item Sia $L$ iniettiva: esiste allora la sua inversa, opportunamente ristretta nell'immagine di $L$, $L(V)$; in altre parole, esiste $L^{-1}\colon L(V)\to V$.
			Possiamo definire quindi la mappa duale $(L^{-1})^*\colon V^*\to L(V)^*$.
			Per ogni $v\in V^*$, consideriamo $w\in Z^*$ dato da
			\begin{equation}
				w=(L^{-1})^*(v)=v\circ L^{-1}.
			\end{equation}
			Allora risulta $L^*(w)=v\circ L^{-1}\circ L=v$, che prova che $L^*$ è suriettiva in quanto esiste sempre una controimmagine (che qui è $w$).
		\item Sia $L$ suriettiva, e consideriamo $\alpha\in\Ker L^*$: risulta allora $0=L^*(\alpha)(v)=\alpha\big(L(v)\big)$ per ogni $v\in V$.
			Ma $L(V)$ coincide con $Z$, perciò $\alpha(z)=0$ per ogni $z\in Z$, ossia $\alpha=0$, e ciò prova che $\Ker L^*=0$ ossia $L^*$ è iniettiva.\qedhere
	\end{enumerate}
\end{proof}

\begin{teorema} \label{t:rango-mappa-duale}
	Data l'applicazione lineare $T\colon V\to Z$ tra gli spazi vettoriali $V$ e $Z$, e la sua duale $T^*\colon Z^*\to V^*$, sia $\tilde{T^*}$ la sua restrizione a $T(V)^*$, ossia $\tilde{T^*}\colon T(V)^*\to V^*$.\footnote{
		Risulta che $T(V)\le Z$ e $T(V)^*\le Z^*$.
	}
	Allora
	\begin{equation}
		\tilde{T^*}\big(T(V)^*\big)=T^*(Z^*),
	\end{equation}
	per ogni $T\in\lin(V,Z)$.
\end{teorema}
\begin{proof}
	Sia $\{e_1,\dots,e_n\}$ una base di $V$ e $\{g_1,\dots,g_m\}$ una base di $T(V)$, che possiamo completare a una base di $Z$ come $\{g_1,\dots,g_m\}\cup\{f_1,\dots,f_{m'}\}$ (perciò $m+m'=\dim Z$).
	Sia dunque $\alpha\in Z^*$: definiamo $\tilde{\alpha}\in T(V)^*$ come $\tilde{\alpha}(g_k)=\alpha(g_k)$ per $k=1,\dots,m$, in modo che $\alpha=\tilde{\alpha}$ in $T(V)^*$.
	Allora $T^*(\alpha)\in Z^*$ e risulta
	\begin{multline}
		T^*(\alpha)(e_i)=\alpha\big(T(e_i)\big)=\alpha\bigg(\sum_{k=1}^m\lambda_kg_k\bigg)=\sum_{k=1}^m\lambda_k\alpha(g_k)=\\=
		\sum_{k=1}^m\lambda_k\tilde{\alpha}(g_k)=\tilde{\alpha}\big(T(e_i)\big)=T^*(\tilde{\alpha})(e_i)=\tilde{T^*}(\tilde{\alpha})(e_i)
	\end{multline}
	e poich\'e $\tilde{T^*}(\tilde{\alpha})\in\tilde{T^*}\big(T(V)^*\big)$ si ha $T^*(Z^*)\subseteq \tilde{T^*}\big(T(V)\big)$.

	Sia ora $\gamma\in T(V)^*$: estendiamolo ad un funzionale di $Z^*$ tramite gli elementi $f_i$ della base di $Z$, ossia definiamo $\tilde{\gamma}\in Z^*$ come $\tilde{\gamma}(g_k)=\gamma(g_k)$ per $k=1,\dots,m$ e $\tilde{\gamma}(f_k)=0$ per $k=1,\dots,m'$.
	Allora, dato che $T(e_i)\in T(V)$, si ha $\gamma\big(T(e_i)\big)=\tilde{\gamma}\big(T(e_i)\big)$ per ogni $i=1,\dots,n$, perciò
	\begin{equation}
		\tilde{T^*}(\gamma)(e_i)=\gamma\big(T(e_i)\big)=\tilde{\gamma}\big(T(e_i)\big)=T^*(\tilde{\gamma})(e_i)\in T^*(Z^*)
	\end{equation}
	da cui segue $\tilde{T^*}\big(T(V)^*\big)\subseteq T^*(Z^*)$, ma allora i due insiemi coincidono.
\end{proof}
\begin{definizione} \label{d:rango-applicazione-lineare}
	Si dice \emph{rango} di un'applicazione lineare la dimensione della sua immagine.
\end{definizione}
Il rango di un'applicazione $T$ è indicato solitamente con $\rk T$, dall'inglese \textit{rank}.
In qualsiasi base, esso coincide con il rango della matrice associata (perciò usiamo lo stesso nome): esso si può dunque calcolare come il numero di colonne linearmente indipendenti di tale matrice.
\begin{corollario} \label{c:rango-righe-colonne}
	Dati due spazi vettoriali $V$ e $Z$ sul medesimo campo, per ogni $T\in\lin(V,Z)$ si ha $\rk T=\rk T^*$.
\end{corollario}
\begin{proof}
	Poich\'e $T\colon V\to T(V)$ è (ovviamente) suriettiva, l'applicazione $\tilde{T^*}\colon T(V)^*\to V^*$ è iniettiva per il teorema \ref{t:iniettivita-suriettivita-mappa-duale}.
	Per il teorema precedente (\ref{t:rango-mappa-duale}) e il corollario del primo teorema dell'isomorfismo segue allora la catena di uguaglianze
	\begin{equation}
		\dim T^*(Z^*)=\dim\tilde{T^*}\big(T(V)^*\big)=\dim T(V)^*-\dim(\Ker\tilde{T^*})=\dim T(V)^*=\dim T(V)
	\end{equation}
	dato che $\Ker\tilde{T^*}=\{0\}$, essendo $\tilde{T^*}$ iniettiva.
	In altre parole, la dimensione dell'immagine di $T^*$ concide con la dimensione dell'immagine di $T$, ossia $\rk T^*=\rk T$.
\end{proof}
Per l'uguaglianza tra il rango dell'applicazione e il rango della matrice associata, questo corollario mostra che il rango calcolato ``per colonne'' è uguale al rango calcolato ``per righe'': in altre parole, il numero di colonne linearmente indipendenti di una matrice coincide con il numero di righe linearmente indipendenti.

